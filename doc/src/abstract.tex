\cleardoublepage
\rm
\begin{center}
	\textcolor{gray} {\large
		An Accelerator Framework for Partially Reconfigurable\\
		FPGA SoCs under GNU/Linux\\
	}

	\vspace{20pt}
	\textsc{\large Abstract}\\
	\vspace{0pt}
\end{center}
The FPGA is an ideal device for hardware acceleration of custom algorithms,
offerring infinite reprogrammability from the end user. Partial Reconfiguration
takes this to a step further, allowing reconfiguration of a part of the device
during run-time, without disturbing the overall system operation.

This technology enables the possibility of a system the provides several
slots for hardware accelerators, that can be loaded on-demand during run-time.
This work implements such a system on an embedded environment, using
the ARM-based FPGA SoCs from Xilinx. The system is built around the AXI interconnect.
It consists of three parts:

\begin{itemize}
\item	Three hardware designs; one homogeneous and one heterogeneous accelerator system for Zynq-7000
	and one for Zynq UltraScale+.
\item	A Linux kernel driver for supporting any hardware accelerator system design under its specifications,
	as well as a system library that provides a user-friendly API for managing the accelerators.
\item	An example application in digital image processing, offering several accelerators for system evaluation.
\end{itemize}

The system is very flexible in supporting hardware designs that may feature accelerator slots of different sizes
and of restricted view to specific memory resources of unequal proximity to the programmable logic.
It offers high performance by using multiple processing system to programmable logic interfaces, balancing
the traffic at the fixed interconnect. It is sufficiently configurable, by allowing a relative preference
of a memory resource over the others, by defining accelerator affinity to hardware slots,
and by providing several scheduler options for slot selection. Finally, it is portable, by decoupling 
the software from hardware, by allowing the use of any supported hardware design provided that is
properly described to the software.

\vspace*{\fill}

