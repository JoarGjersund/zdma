\cleardoublepage
\rm
\begin{center}
	\textcolor{gray} {\large
		An Accelerator Framework for Partially Reconfigurable\\
		FPGA SoCs under GNU/Linux\\
	}

	\vspace{20pt}
	\textsc{\large Abstract}\\
	\vspace{0pt}
\end{center}

Currently there is a rising trend in accelerating specialized computation,
mostly driven by the growing need for machine learning. The solutions can
be classified to two major groups: novel processor architectures that are tailored
for the targeted computation problem and hardware acceleration, which is represented
by ASICs and FPGAs. The advantage of lower non-recurring engineering and field reconfiguration
gave FPGAs a dominant position in the second group. Partial reconfiguration further extends
the FPGA reconfigurability by allowing the reconfiguration of a part of the device
during run-time without interrupting the overall system operation.

This technology makes possible to create a system that offers mutliple accelerator cores
which can be reconfigured on-demand during run-time.
This work implements such a system using the Zynq SoC from Xilinx, using the AXI as data transport.
It consists of following parts:

\begin{itemize}
\item	Hardware implementation of one homogeneous and one heterogeneous accelerator system for Zynq-7000 SoC,
	as well as a port to the newer and more advanced Zynq UltraScale+.
\item	A Linux kernel driver for supporting any hardware accelerator system design under its specifications
	and offers an abstracted view of the system to the user applications.
\item	A system library that provides a user-friendly API for managing the accelerators.
\item	An application in digital image processing that implements 
	several common accelerators, for system evaluation.
\end{itemize}

The system achieves high performance by taking advantage of the SoC interconnect and 
balances data traffic among the ports that connect the FPGA fabric to the SoC.

It supports memory segmentation so a user can define multiple memory resources that can be
different hardware devices and/or are connected at a varying proximity to the processor, all being accessed concurrently.
The user may additionally express their relative preference of a memory resource versus the others,
matching the capability and shaping the traffic of the corresponding underlying interconnect.
Accelerator access to memory can be restricted to offer isolation for security or quality-of-service. 

The user can define a desired set of accelerator core slots where their task is allowed to execute,
whereas the administrator is able to define which accelerator variants can be programmed to each slot.
In this way performance can be maximized if the workload can be predicted.

The accelerator slots can be unequal and an accelerator variant may therefore not be placeable in every slot.
In this way we can create a heterogeneous system of slots with different size 
or containing different specialized fixed logic.

Finally, the system was designed to be as flexible and portable as possible.
Although a few designs were implemented, they act as examples --
the designer can implement any hardware configuration within specs
with the only obligation to properly describe it in a textual file that is passed to the kernel.
None of this information is exported in the userspace so the software applications have only
an abstracted view of the acceleration capability.

\vspace*{\fill}

