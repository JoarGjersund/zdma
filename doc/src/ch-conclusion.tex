\chapter{Conclusion and Future Work}

We developed a complete reconfigurable system, capable of providing hardware accelerators on-demand to the end user.
The system was designed with a server environment in mind -- that is, a capable GNU/Linux system performing
random user tasks in an unpredictable timing but of somewhat predictable nature, whose computational kernel
can be off-loaded to the FPGA. As such, it is capable of handling any dynamic load within the system's memory limits,
with no pre-defined static schedule or any other prior knowledge.

Usage scenarios may include a web server, which can use the FPGA to perform the common encryption/decryption and
compression/decompression tasks that are typical to such a workload. Our specific application could be made useful
in a server that performs image or other media processing over a vast number of user content. Finally, it could
also find place in an interactive multimedia application,
allowing high-performance video editing or other media transformation.
If the user-level application can tightly control the timing of execution requests, this system could even serve
as a backend to a higher level Real-Time Management System.

The design of the system tried to cover many possible scenarios, including the need of accelerators with
a dedicated path to the memory, accelerators of different degree of criticality,
heterogeneous memory resources of different speed, etc.
It is designed right from the start with concurrent memory access in mind and it takes advantage of the
platform hardware capabilities for parallelization. Acknowledging the variety of usage scenarios, several
scheduling and accelerator replacement algorithms were implemented.
Additionally, the administrator may define an affinity mask of the accelerator variants to the available slots,
wheras the user may define a similar mask for their tasks. This way it is possible to enforce isolation,
either for quality of service and predictability or for security.

A system shared library is provided that abstracts all system details, making it easy of the application
developer to use the system without any need of hardware knowledge.

The framework was successfully implemented in the Zynq-7000 based Zedboard from AVNET.
Two different architectures were realized in order to assess design parameters such as the
interconnect architecture, the memory layout, the accelerator data width and pipeline depth, etc.

A digital image processing application was implemented and used as a benchmark. 
Certain interesting observations were made from which we can deduct the validity of our design choices.

An attempt to port the system to the UltraScale+ based zcu102 was thwarted at the software porting phase,
as Xilinx had yet to provide required software support for Partial Reconfiguration on this device class.
Nonetheless, the hardware design was complete, and this allowed us to assess the new architecture's
capabilities, including the complete overhaul of partial reconfiguration support.

\section{Challenges and Lessons Learned}

The development of this system was not without unpredicted or underestimated problems. In this section
we will try to discuss some of the noteworthy issues that arose during development.

\subsection{The Implementation Workflow}

The first and foremost is our underestimation of the difficulty that arises with the hardware implementation 
of such a system and of partial reconfiguration in general. 
A basic working system to be used as a proof-of-concept is rather easy to implement.
However, as the clock frequency and the number of \glspl{rp} rise, things change a lot. A major issue is that there is
no partition placer tool despite that proper size, shape and placement is critical for attaining the highest perfomance.
Thus, the work of the placer had to be done manually. The effort of this endeavor increases exponentially with the target clock.
For example, our 16-core design could possibly be successfully routed with the first attempt at 100MHz. Increasing the clock
to 125MHz requires many hours or tweaking. In order to reach the final goal of 133MHz it required several days of experimentation
with partition moving or swapping, changing tool settings, chosing the correct initial configuration, finding per-variant
optimal implementation settings, etc. As performance difference proved to be minimal, it is a question if it was worth the effort.

%Needless to say that before attempting partial reconfiguration, the designer must ensure that at least the static logic and
%the module that will consist the initial configuration are verified to conform to specifications in a normal workflow.
%The slightest change in the design might require comparable effort to the inital implementation.

The partition sizing was a decision that if it was to be taken again, it would be taken differently.
In section \ref{sec:sizing} we discussed the sizing trade-offs. A factor not properly weighed was the
availability and capability of routing resources. Each partition comes with a fixed cost which does
not constitute only the LUTs of AXI DMA, but also the wires that connect it, the interconnect crossbar,
the wires that implement the control interfaces of both the accelerator and the AXI DMA, as well as their own,
seperate interconnects. Routing can become difficult and at some point, it will necessitate a clock frequency reduction.

\subsection{The Tool Quality}

An important issue proved to be the implementation tool maturity. The partial reconfiguration TCL scripts that Xilinx offers
are not optimized at all. They are not multithreaded even in cases it would be easy to do so, e.g. a parallel \gls{ooc} synthesis
of module variants or parallel implementation of module variants. Some work could be reused, e.g.
the implementation of each variant needs carving out the modules of initial configuration,
and this is repeated for each variant when the carved configuration could be saved and reused. 

As of version 2017.2,
the Vivado implementation tool support for UltraScale devices appear to be immature. Partial reconfiguration frequently
causes locking of special resources (BRAM and DSP tiles) during initial configuration, causing the placement failure of the next.
This is irrespectively of the variant used for each configuration. The issue was worked around by reshaping the affected partitions,
but no golden rule could be figured out for what consists a favorable shape, and no help could be gained from Xilinx forums.

A final and very important issue is the tool speed. At the GUI front, designing larger systems visually is virtually impossible.
The slightest action causes a sequence of object property propagation which can be an excruciating for a large design.
This can manifest at some unexplainable ways, like refreshing the custom IP directory data when the user changes the floorplanner's color palette.
The designer must not hesitate to learn operating Vivado from the TCL console -- the effort invested will pay out fast.

At the backend front, the workflow on ZynqMP target is sluggish in every possible way. Implementation is much more time consuming
that the four-fold increase of LUT count between Z-7020 of Zedboard and XCZU9EG of zcu102 could justify. Even marking a module
as a black-box is considerably slower; a process that takes less than a minute on Z-7020 would take more than an hour on XCZU9EG.
A full partial reconfiguration synthesis, implementation, verification and bitstream generation would need several hours at most
for the former but around a week for the latter in a dual Xeon X5660. 

Clearly, in any new project an UltraScale-class device should not be used for initial development.
It would be preferrable that first stages to be done in a 7-class device if possible,
and then switch to an UltraScale-class device when the design requires device-specific development.

\subsection{The Efficiency of HLS}

Finally, the usefulness of the HLS tool should be reconsidered. The HLS was chosen for accelerator implementation as it
allows quick prototyping.
The time from the initial algorithm conception to a validated working bitstream can be more than an order
of magnitute less than of a traditional HDL workflow. 

However, from that stage to the final optimized version is a completely different story.
The resource consumption and attained latency of HLS is difficult to control and even more difficult to predict.
As we saw at section \ref{sec:gaussian}, a minor chage of filter values reduced overall LUT utilization by 11\%.
%In table \ref{tab:acc-util}
%we saw that the emboss filter has twice the latency of the sharpen filter, despite that these filters have identical complexity.
%Oddly enough, it also has twice latency compared to sobel which has a two-fold increased complexity.
%TODO na to tsekarw prota

Extreme caution should be paid to variable sizes as they dictate the inferred execution unit size -- e.g.
in contrast accelerator (see \ref{sec:contrast}) we used \texttt{int} types to express the brightness and contrast parameters, 
as from the processor we only do 32-bit load/stores.
This inadvertently later caused a 32-bit divider to be inferred when an 8-bit would suffice. In many cases, when the bounds of the values
a variable will receive in its lifetime can be known at compile-time, HLS will trim the variable accordingly -- a typical example
is the index of a constant bounds loop. Integer promotion oftentimes may be inconsistent.
For example, if a product of two \texttt{short}s (16 bits) is stored to a \texttt{char} (8 bits), an 8-bit multiplier will be inferred.
However, if operands were \texttt{int}s (32 bits), a 32-bit multiplier would be inferred. Usually these discrepancies will be sortened out
by the synthesizer. Finally, the effect of setting the clock target can be odd. If HLS fails to achieve the desired period,
one could try an even more difficult target and the HLS may restructure the code to achieve the original, previously failed, target.

To make things worse, the HLS resource utilization and latency estimates should be treated as randomly generated numbers.
In order to confirm resource utilization, the designer should export the compiled output to a synthesized device checkpoint.
The post-synthesis resource utilization report is far more precise. Regarding latency, numbers may be deceptive. For example,
in figure \ref{fig:acc-throughput} we see that the measured throughput does not match the HLS latency estimation shown in table \ref{tab:acc-util}.

%Some of these issues could possibly consist current implementation weaknesses that could be mitigated in the future.
%However there have been difficulties that arise from the language's design. 
%C and C++ were never designed as anything but software languages and therefore all their operators reflect common CPU operations
%and do not possess the expressivity to describe common hardware concepts. Xilinx has made the effort of extending the language
%to offer some essential hardware concepts, most importantly the arbirtrary integer arithmetic. Some 

%Overall, the effort spent trying to control resource utilization and latency was multiple orders of magnitude higher than
%what the initial working version needed. Certainly, even more is required to achieve optimal performance -- for example,
%the accelerators that perform 2D convolution trail in latency at 8-to-1 ratio compared to the ones that peform pixel transformations,
%with suspected culprit the inefficent use of the BRAM by the linebuffer class.

At the end of the day, the use of HLS was a regretful decision.
The effort spent to control resource utilization and latency was unexpectedly high and the result was mediocre.
It was viewed as a quick and easy solution to implement a simple test application to evaluate our system 
but it proved to be neither easy, and certainly, nor quick.

\section{Future Work}

The are several opportunities to optimize or extend this work.
They range from simple additions and improvements to 
revising significant architectural decisions.
In this section, we will discuss some of these ideas.

\subsection{Integration with FPGA Manager}

We discussed FPGA Manager in section \ref{sec:linux-pr}. 
It is imperative to replace the partial reconfiguration driver from \texttt{devcfg} for two main reasons:

\begin{itemize}
\item	\textbf{Portability}: FPGA Manager is a Linux kernel API and that offers hardware abstraction to
	FPGA manipulation.
	Using it, it will not only provide future support for ZynqMP, but it will allow porting the system
	to other architectures like Microblaze or even to Altera/Intel SoCs.
\item	\textbf{Integration}: Making Linux to be aware of the FPGA is a big step forward.
	We already discussed the advantages of this framework but we should stress again that this is not just
	an FPGA programming interface. 
	As the framework matures, more and more FPGA related functionality will be added, 
	obsoleting custom and out-of-tree solutions.
\end{itemize}

If that reasoning does not sound compelling enough, 
we should add that Xilinx has stopped development of \texttt{devcfg} and has 
completely removed it in its latest kernel (v4.14).

\subsection{Use of Advanced DMA Modes}

The AXI DMA IP core was used in Direct Register mode because it is the simplest solution, not the most efficient.
The drawback of this method is that there is a significant time interval between 
one transaction completion and next transaction start, during which the controller is idling. 

The interrupt service latency should not be blamed as it is usually at the 100μs range,
a value quite small compared to the transaction processing latency or the reconfiguration time,
both at the order of a milliseconds. The problem arises from the fact that we do not immediately
know what transaction is going to be processed afterwards. The corresponding slot will be marked as free,
and it will wait to be chosen at a later scheduler invocation. How long this will be, depends on the system
load and how frequently the Linux process scheduler is called, which is configurable.

Even if we manage to quantify and/or reduce it, we still need to analyze whether it is avoidable.
If the system includes many accelerators and few slots, it is highly probable that the slot that has just finished
will be reconfigured, which closes the door for any optimization. On the contrary, if there is a good probability
this slot will re-use the same accelerator, it would be very beneficial if we could immediately schedule next transaction.

The Linux DMA Engine API supports scatterlists, an abstraction of DMA Scatter-Gather.
It is a software construct; an array of transaction descriptors that the API uses
the program the DMA controller, to avoid the unnecessary controller pauses.
It does not reqire hardware support, but if present, it will be taken advantage of.
The AXI DMA ip core does support scatter-gather,
but at a cost of an one-third increase of resource utilization (see table \ref{tab:dma-modes}).

Regardless from whether it is beneficial in most situations, the driver should provide this possibility.

\subsection{Rethinking the Accelerator - DMAC Relationship}
\label{sec:future-dmac}

Initially, the system was envisioned to be capable of supporting a handful of accelerator slots, with four being the initial target.
With that in mind, using one AXI DMA for each accelerator slot was not an issue. An assumption that may still hold true for the 6-core design,
but was first challenged at the 16-core design. When the capacity of the XCZU9EG permitted the implementation of 63 accelerator slots,
it cast a doubt on whether 63 AXI DMA controllers were really necessary.

There are two reasons that drive us to object the current design:

\begin{enumerate}
\item	An accelerator is not always capable of pulling an AXI \gls{beat} per cycle. 
	Our convolutional filters pull data once per 8 cycles, so the AXI DMA IP is forced to wait for 7 out of every 8 cycles.
	If we could time-share the DMA controller, we could theoretically support 8 accelerators from a single DMA controller.
\item	It is common that a single compuational task is consisted of multiple stages of processing, each being represented as an
	accelerator core. Instead of pushing in and out data from and to every accelerator, each accelerator could drive directly
	the next stage. If we could do this, we would have another major gain: the data stream would flow directly within the PL realm,
	without reaching main memory until the last link of the processing chain has finished. This would greatly reduce latency
	and relieve the pressure on the memory controller.
\end{enumerate}

For the first case, we can propose exploring Multi-Channel DMA operation. In this configuration, a DMA controller can control
multiple AXI streams from a single interface, by time-sharing its resources. The streams can then be routed to their correct
destination after setting the user AXI signal \texttt{TDEST}. The AXI DMA IP has this capability and recently Xilinx released
the AXI MCDMA IP core. The routing can be performed using an AXI Stream Switch. Of course, both the mutli-channel support and
the routing take up FPGA resources. For the former, the table \ref{tab:dma-modes} can be advised, and as for the latter
one may refer to \cite{pg085}. Despite the idea is to share a DMA controller among multiple accelerators, the same
principle can also be aplied in case we would need multiple input or output streams for a single accelerator.

The second case is more complex and the ideas behind it may be materialized in several ways, depending the target application.
The possible approach would be to chain accelerator slots. It could be implemented either as a set of peers or as a big main
core, accompanied by smaller pre and post processing cores. In any case, the output stream of the one core is driven to the
input stream of the other. It would be a direct connection, no interconnect is necessary. If a core within a chain is not needed,
it will be loaded with a loopback module.

One could object that in a scenario that we have few and large accelerator slots serving tasks that cannot be easily broken down
to elementary computations, such an arrangement would lead to far too many wasted FPGA resources by cores loaded with a loopback module.
We could withdraw our idea of having only a single DMA controller but keep our idea to directly connect cores, just to avoid
unnecessary main memory access. We could even extend it, to include the possibility of sourcing or sinking data from/to a
hardware peripheral, to support a real-time embedded setting.

\begin{figure}[tb!]
\begin{subfigure}[h]{\textwidth}
\centering
\begin{tikzpicture}
	% The Accelerators
	\node[block, minimum width = 60mm, minimum height = 16mm] (axidma)
		at (0,0) {MC AXI DMA};

	\node[block, minimum width = 25mm, minimum height=8mm, above=10mm of axidma] (acc0) {Accelerator 3};
	\node[block, minimum width = 25mm, minimum height=8mm, above=3mm of acc0] (acc1) {Accelerator 2};
	\node[block, minimum width = 25mm, minimum height=8mm, above=3mm of acc1] (acc2) {Accelerator 1};
	\node[block, minimum width = 25mm, minimum height=8mm, above=3mm of acc2] (acc3) {Accelerator 0};

	\node[interconnect] at ($(acc1)!0.5!(acc2) + (-25mm, 0)$) (int0) {};
	\node[interconnect] at ($(acc1)!0.5!(acc2) + ( 25mm, 0)$) (int1) {};
	\node[textonly, left of=int0,xshift=-2mm] {\rotatebox{90}{1MI x 4SI}};
	\node[textonly, right of=int1,xshift=2mm] {\rotatebox{90}{4MI x 1SI}};

	\coordinate (p0) at ($(int0.north east)!0.20!(int0.south east)$); \draw[axis64] (p0) -- +(2.5mm,0) |- (acc3);
	\coordinate (p1) at ($(int0.north east)!0.40!(int0.south east)$); \draw[axis64] (p1) -|- (acc2);
	\coordinate (p2) at ($(int0.north east)!0.60!(int0.south east)$); \draw[axis64] (p2) -|- (acc1);
	\coordinate (p3) at ($(int0.north east)!0.80!(int0.south east)$); \draw[axis64] (p3) -- +(2.5mm,0) |- (acc0);
	\coordinate (q0) at ($(int1.north west)!0.20!(int1.south west)$); \draw[raxis64] (q0) -- +(-2.5mm,0) |- (acc3);
	\coordinate (q1) at ($(int1.north west)!0.40!(int1.south west)$); \draw[raxis64] (q1) -|- (acc2);
	\coordinate (q2) at ($(int1.north west)!0.60!(int1.south west)$); \draw[raxis64] (q2) -|- (acc1);
	\coordinate (q3) at ($(int1.north west)!0.80!(int1.south west)$); \draw[raxis64] (q3) -- +(-2.5mm,0) |- (acc0);

	\draw[raxis64] (int0) -- (int0 |- axidma.north)
		node[below]{\footnotesize\ttfamily MM2S};
	\draw[axis64] (int1) -- (int1 |- axidma.north)
		node[below]{\footnotesize\ttfamily MM2S};
\end{tikzpicture}
\caption{Multichannel DMA}
\end{subfigure}
\bigskip\\
\begin{subfigure}[h]{.4\textwidth}
\centering
\vspace{6.5mm}
\begin{tikzpicture}
	% The Accelerators
	\node[block, minimum width = 50mm, minimum height = 16mm] (axidma) at (0,0) {AXI DMA};

	\node[block, minimum width = 15mm, minimum height = 16mm, above=25mm of axidma] (acc) {Acc. 1};
	\node[block, minimum width = 15mm, minimum height = 16mm, left=5mm of acc] (pre) {Acc. 0};
	\node[block, minimum width = 15mm, minimum height = 16mm, right=5mm of acc] (post) {Acc. 2};


	\coordinate (y0) at ($(acc)!0.5!(axidma)$) ;
	\coordinate (x0) at ($(acc.south west)!0.25!(acc.south east)$) ;
	\coordinate (x1) at ($(acc.south west)!0.75!(acc.south east)$) ;

	\draw[axis64] (pre |- axidma.north) node[below]{\footnotesize\ttfamily MM2S} -- (pre);
	\draw[axis64] (pre) -- (acc);
	\draw[axis64] (acc) -- (post);
	\draw[raxis64] (post |- axidma.north) node[below]{\footnotesize\ttfamily S2MM} -- (post);
\end{tikzpicture}
\caption{Chaining Accelerators}
\end{subfigure}
\begin{subfigure}[h]{0.6\textwidth}
\centering
\begin{tikzpicture}
	% The Accelerators
	\node[block, minimum width = 55mm, minimum height = 16mm] (axidma) at (0,0) {AXI DMA};

	\node[block, inner sep=0pt, minimum width = 20mm, minimum height = 16mm, above=25mm of axidma] (acc) {Accelerator};
	\node[interconnect, left=5mm of acc] (pre) {};
	\node[interconnect, right=5mm of acc] (post) {};


	\coordinate (y0) at ($(acc)!0.5!(axidma)$) ;
	\coordinate (x0) at ($(acc.south west)!0.25!(acc.south east)$) ;
	\coordinate (x1) at ($(acc.south west)!0.75!(acc.south east)$) ;

	\draw[axis64] (pre |- axidma.north) node[below]{\footnotesize\ttfamily MM2S} -- (pre)
		node[below left=5mm and 0mm, text width=18mm, align=right]{\scriptsize\ttfamily from main\\\vspace{-3mm}memory};
	\draw[axis64] (pre) -- (acc);
	\draw[axis64] (acc) -- (post);
	\draw[raxis64] (post |- axidma.north) node[below]{\footnotesize\ttfamily S2MM} -- (post)
		node[below right=4mm and 0mm, text width=18mm, align=left]{\scriptsize\ttfamily to main\\\vspace{-3mm}memory};

	\draw[axis64] (post.east) -- +(10mm, 0) node[above, text width=18mm, align=left]{\scriptsize\ttfamily next\\\vspace{-3mm}accelerator};
	\draw[axis64] (post.north) -- +(0mm, 10mm) node[below right, text width=15mm,align=left]{\scriptsize\ttfamily sinking\\\vspace{-3mm}peripheral};
	
	\draw[raxis64] (pre.west) -- +(-10mm, 0) node[above, text width=18mm, align=right]{\scriptsize\ttfamily previous\\\vspace{-3mm}accelerator};
	\draw[raxis64] (pre.north) -- +(0mm, 10mm) node[below left, text width=15mm,align=right]{\scriptsize\ttfamily sourcing\\\vspace{-3mm}peripheral};

\end{tikzpicture}
\caption{Routing streams to/from other hardware}
\end{subfigure}
\caption{Alternative accelerator and DMA controller arrangements.}
\label{fig:alt-arrangements}
\end{figure}

\subsection{Task Buffer Migration}

The system allocates a buffer pair each time the user application configures a task. If a configured task
declares a new configuration, the old buffers are released and new buffers, of the new requested size will be allocated.
The new buffers will be placed wherever the memory allocator deems best -- there is no connection to the old buffers,
and data will not be preserved. 

Since this action is initiated by the user, the data loss is acceptable since we assume the user will only issue such a request
when they have finished processing and/or saved their data. However, there might be cases it would be beneficial to perform
an involuntary buffer move. For example, the external fragmentation of a memory zone might not permit a creation of a new
task despite that there is enough free space. In another case, there might be free and contiguous space in a zone where 
a task in not allowed to be placed due to hardware or affinity restrictions. 
Finally, a re-arrangement of user buffers among memory zones may be help re-balance data traffic to maximize parallellization.

Since these scenarios are determined by the kernel without the consent of the user application, they have to be made transparently
and without data loss. Indeed, a key reason we implemented a custom memory allocator was to make this feature easier and simpler
to implement. Implementation should be straiforward:

\begin{enumerate}
\item	Ensure that no DMA transaction is currently taking place on behalf of this task 
	and prevent this from happening while the migration takes place.
\item	Find the new optimal buffer location.
\item	Update allocator bitmap to reserve the new space.
\item	Migrate the page frames to the new location, updating the user process page table.
\item	Update allocator bitmap to release the old space.
\end{enumerate}


\subsection{Accelerator Control}

The \gls{axilite} protocol rightfully gained its name as it requires only a few dozen of LUT instances to implement a slave interface.
Nonetheless, this slave interface consists of 94 wires. Despite that we have so modest demands for throughput from a configuration
port, the interface still implements seperate address, read and write channels, as all members of the \gls{axi} family do.
On top of that, we should add the implementation of the required interconnect.

It is worth exploring the use of a simpler protocol, even a serial one. The I\textsuperscript{2}C is rather slow but it uses only
two wires (clock and data) shared between all slaves. The SPI needs a clock, a data-in and a data-out plus one slave select per slave,
but it is significantly faster.
A more efficient solution would be to embed configuration data in the data stream. It adds some implementation complexity
and might be a bit restrictive in some cases, but it comes at no cost in routing resources.

Moreover, currently, the accelerator control is modeled after the Vivado HLS control signals. 
While it would not be very difficult to generate these signals from an HDL or another HLS tool,
a cleaner solution would be to define a custom interface and offer wrappers for any HLS tool.

\subsection{Accelerator Interrupts}

We assumed an accelerator would have a input and an output stream, with no other side-effects.
However, an accelerator may need to generate interrupts. Support of this feature may take the form of
offering a system call that blocks until the interrupt is acknowledged, as it is done in EPEE\cite{epee}.
Another approach would be by delivering POSIX Signals, in order to allow the user application to perform
useful work while waiting for the notification.

\subsection{Portability}

The system was designed using the Xilinx tools on a Zynq platform. As it is already discussed, FPGA Manager will
solve the unportable nature of partial reconfiguration. 

Two additional points that tie us to a specific architecture are the AXI interconnect and the ARM processor.
The former is not expected to be an issue, at least if we fix the accelerator control protocol (see above).
The latter is a bit more a senstive issue. While we do not bind to the processor architecture, we extensively use
the \gls{ftd}. GNU/Linux has decoupled it from \gls{of} and the current trend is to expand support to more platforms.
Despite that support for x86 was introduced back in 2011, it is uncertain how easy would be the transition to
the future x86-connected FPGAs.

However, the biggest -by far- burden is the heavy dependence on Xilinx IP cores. It is certain that
Intel/Altera would have an equivalent core for every basic core of Xilinx, however they would differ on
configuration and interface.

While it is unavoidable to get rid of all of them, as some provide support for vendor specific hardware,
they could be reduced to a minimum by replacing generic functionality (like a DMA controller) with
other academic and/or open-source alternatives.

\subsection{Scheduler Improvements}

So far, the scheduler may decide only by taking into account the past (frequency of use, time elapsed from an event)
or a user wish (e.g. priority, affinity). The scheduler has no knowledge of the future except from
the tasks that are in-flight trying to find a free core.

A useful feature would be to help the scheduler by stating an intent to execute a certain list of tasks.
This way it can collect information that will more accurately predict which accelerator core would less likely be needed
or it could postpone or re-arrange scheduling of tasks so to minimize the number of slot reconfigurations.

\subsection{Bitsteam Size}

In our work, for each module variant we produced a partial bitstream for every \gls{rp}.
This was a significant burden for the memory constrained Zedboard. We did use XZ compression for bitstream storage,
but we chose to decompress them in the kernel during registration time, as doing so on-demand would incur a penalty of
more than 25ms. One might wonder, why not just use one for every variant?

In modern FPGAs, the routing of the static design is permitted to cross a \gls{rp} in order to reduce pressure on router.
This had as a consequence that every partial bitstream is totally different for each \gls{rp} it is implemented.
Xilinx does not provide a way to restrict this routing apart from constraining all static routing to a static pblock,
with whatever impact it would have on the quality of the result. The recent Altera Stratix 10 has a different architecture
what permits bitstream relocation. For a detailed discussion on bitstream relocation and other current PR-related topics,
one can address to \cite{survey}.

There are several attempts to tackle this issue, either with run-time relocation or with bitstream manipulation prior programming.
At bare minimum, the native bitstream compression should be attempted. Since it is processed directly from the hardware,
the latency overhead could be small.

\subsection{Clock Management}

The system was not designed with power consumption in mind. There can be found opportunities for lowering the power envelope.
For static power, there is little we can do. The Zynq platform does have programmable voltage regulators but they are global to the PL
and therefore we cannot power down a single \gls{rp}.

However, it is possible to reduce dynamic power consumption. Even without an external clock source, Zynq and ZynqMP can create and distribute
four configurable clocks. Either by delivering multiple pre-set clocks or by using clock converters, it would be possible to offer more than
one clock to an accelerator slot. 

Depending user desire, policy or simply wheter system is operating from battery, the clock source could be switched to save energy.
Furthermore, it would be beneficial for more complex accelerator designs that may fail timing closure at maximum clock
to be given the oportunity to run at a slower one.

\subsection{Extend Heterogeniety}

In our work, support for heterogeniety is minimal. 
It is simply offered with the notion that if a certain module variant doesn't fit, it's ok, we'll schedule the task somewhere else.

This model has to be extended and the scheduler must be aware of the slot size and the special resources it contains.
We already take into account the slot parameters for defining a slot attractiveness, but if the scheduler had more information
about the slots and/or the task demands, it could even affect its decision taking into account time slacks, power consumption,
and even chose an appropriate slot data width.


\subsection{Extending the Streaming Model}

We used the streaming model under the assumption it is the most appropriate for an FPGA, 
as they typically lack sufficient memory resources.

However, Xilinx recently presented UltraRAM, an on-chip memory resource for the UltraScale+
that offers significantly higher capacity but lower performance compared to BlockRAM. 
Additionally, we should add that a designer may implement a memory controller in the programmable logic.
Xilinx offers the ``Memory Interface Generator'' IP core to create such an interface to any supported FPGA.
Altera/Intel offers hard blocks for memory controllers in their Arria V SoCs.
As the manufacturers advance their capabilities in 2.5D CMOS, it reasonable to assume that FPGAs will
gain access to large on-chip memories. 

Support for a random-access co-processor model will become significant. 
Furthermore, cache-coherency is possible and should be offered in such a model. 
For a discussion on cache-coherence in Zynq and ZynqMP, please see section \ref{sect:interconnect}.

\subsection{Scaling to Multiple Boards}

A very interesting possibility for exploration would be extending the system to more than one FPGA.
A master-slave architecture could allow the master Zynq to utilize the PL of other boards, not necessarily equipped with a SoC.
Essentially all control and main memory resides in a single node whereas all other FPGAs would be used to stream data to, process,
and stream back the output to main node.
If all boards are Zynq, a peer-to-peer architecture could be implemented, where any Zynq could reserve and utilize an accelerator
slot from any other Zynq.

Exchanging control signals throughout all the boards would be a trivial problem which can be solved even using the FPGA pins exposed at
the Pmod connectors. Higher level connectivity could be used, like Ethernet, but with a significant impact in latency.
The high-speed data transport would be a more complex issue. As a physical layer for the link, single-ended or LVDS I/O pins can be used
but if the FPGA contains serial transceivers they would be a superior choice. An AXI channel can pass over the Aurora link-layer protocol,
which can utilize the FPGA transceivers. Xilinx offers two IPs for bridging an AXI Stream, the Aurora 8B/10B and Aurora 64B/66B.

It is worth to mention that Xilinx also offers the AXI Chip2Chip IP core. This core can utilize the Aurora IPs
to allow briding of the full, memory-mapped AXI. 
This would make possible for a processor to access memory buffers residing in another board, essentially creating a non-uniform 
shared memory system of FPGA nodes. However, this would not be just an extension but a radically new approach.

