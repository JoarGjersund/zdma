\chapter{Application and Evaluation}

In order to demonstrate and evaluate the proposed system, an example application was created.
The chosen field was digital image processing, and a number of common filters were implemented
in Vivado HLS to be used as accelerators.

Vivado HLS is an attempt from Xilinx to use a software language to describe hardware.
It accepts a subset of C/C++ and produces Verilog code, that can be synthesized with
the existing synthesis tools.

The advantage of this approach is that it a allows a very fast production of a working
prototype, in order to assess the feasibility of an idea. Since the code describes
an algorithm function and not its implementation, design exploration can be automated.

There is a price to be paid though, that stems from both the core idea of HLS and its
current incarnation. More on this subject is discussed in the ``Lessons Learned'' chapter.

\section{Accelerator Description}

In image processing, from computational aspect, there are two main categories of filters.
The first, applies a transformation at pixel level. Using a mathematical formula or a
look-up table, it maps one pixel value to another irrespectively of any other factor.
The second category needs spatial awareness -- the new pixel value does not depend only
on its old value, but also on its neighbors. To quantify this dependency, a ``mask'' or
``kernel'' is created, which is usually an odd-dimension square (or less often, rectangular)
matrix whose each value represent the weight of the dependency, with its central value
representing the current pixel. The kernel will scan the whole image pixel after pixel,
and for each pixel it will calculate the sum of the products of the kernel's weights
and the corresponding pixel values. This mathematical procedure is called
``two-dimensional discrete convolution'' and is symbolized with an asterisk.
To summarize,

\begin{figure}[H]
\centering
\begin{subfigure}[b]{.3\linewidth}
	\[
		p'(x,y) = f\left[p(x,y) \right]
	\]
	\caption{Pixel transformation}
\end{subfigure}
~\quad
\begin{subfigure}[b]{.6\linewidth}
	\[
		p'(x,y) = \sum_{m = -M/2}^{M/2}{
			\sum_{n = -N/2}^{N/2}{
				{h_{m,n}\cdot p(x - m, y - n)}
			}
		}
	\]
	\caption{2D Filtering}
\end{subfigure}
\end{figure}

Convolution has a very interesting property: It is corresponds to element-wise multiplication
in the frequency domain. Therefore, we could calculate a discrete Fourier
transform our image and our kernel, after padding the latter to the size of the former,
multiply the transformed images, and apply the inverse discrete Fourier transform
to return to spatial domain. In simpler words,

\[
	p' = IDFT\left\{ DFT\{h\} \cdot DFT\{p\} \right\}
\]

In respect to filter size, the latter method has an $O(1)$ complexity, 
whereas the former has $O(n\cdot m)$. As crossing the spatial and frequency domains
has a computational cost, the former method is cheaper for small $m,n$.
For CPUs, the OpenCV library uses the kernel dimension of $11x11$ as the turning point.

In our work, we use only 3x3 and 5x5 kernels. However, the principal deciding factor
is that the former method has an obvious implementation in a streaming environment
with no random memory access capability like ours. A line buffer that keeps the last $n$ lines
is sufficient\footnote{Theoretically, only a $n-1$ line buffer is necessary but the
implementation would be more complex.}. For a 1080p and a 5x5 kernel, this translates
to a \gls{bram} usage of around 42kibits or 3 BRAM18 instances, which is perfectly affordable.

It should be noted that this method normally requires some special treatment of the
border pixels, as the convolution kernel will attempt to access pixels outside the image.
Several techniques exist, from padding with zeros, mirroring the edges, etc.
For simplicity shake, we do not take care of this. As a consequence, a thin line 
of undefined contents will be visible at the upper and the right border.

For pixel transformations, an actual computation was used instead of look-up tables.
This is because the transformations are either trivial (negative, threshold) or
user configurable (contrast) and it is impractical to create table entries for every possible
user choice.

\begin{figure}[tb!]
\centering
\begin{subfigure}[b]{.49\linewidth}
	\includegraphics[width=\linewidth]{img/loopback.jpg}
	\caption{loopback}
\end{subfigure}
~
\begin{subfigure}[b]{.49\linewidth}
	\includegraphics[width=\linewidth]{img/contrast.jpg}
	\caption{contrast (contrast -32, brightness +32)}
\end{subfigure}
\par\bigskip
\begin{subfigure}[b]{.49\linewidth}
	\includegraphics[width=\linewidth]{img/negative.jpg}
	\caption{negative}
\end{subfigure}
~
\begin{subfigure}[b]{.49\linewidth}
	\includegraphics[width=\linewidth]{img/threshold.jpg}
	\caption{threshold, th=127}
\end{subfigure}
\caption{Pixel Transformations. Test image is in public domain.}
\label{fig:pixel-transformations}
\end{figure}

\begin{figure}[tb!]
\centering
\begin{subfigure}[b]{.49\linewidth}
	\includegraphics[width=\linewidth]{img/gauss.jpg}
	\caption{gauss}
\end{subfigure}
~
\begin{subfigure}[b]{.49\linewidth}
	\includegraphics[width=\linewidth]{img/sharpen.jpg}
	\caption{sharpen}
\end{subfigure}
\par\bigskip
\begin{subfigure}[b]{.49\linewidth}
	\includegraphics[width=\linewidth]{img/emboss.jpg}
	\caption{emboss}
\end{subfigure}
~
\begin{subfigure}[b]{.49\linewidth}
	\includegraphics[width=\linewidth]{img/outline.jpg}
	\caption{outline}
\end{subfigure}
\par\bigskip
\begin{subfigure}[b]{.49\linewidth}
	\includegraphics[width=\linewidth]{img/sobel.jpg}
	\caption{sobel (original kernel)}
\end{subfigure}
~
\begin{subfigure}[b]{.49\linewidth}
	\includegraphics[width=\linewidth]{img/scharr.jpg}
	\caption{sobel (Scharr kernel)}
\end{subfigure}
\caption{Two-dimensional Filtering Kernels}
\label{fig:2d-filters}
\end{figure}

\subsection{Trivial Pixel Transformations}

In total, there are three pixel transformation accelerators:

\begin{itemize}
\item	\textbf{loopback}: $p'(x,y)=p(x,y)$ -- A dummy accelerator that justs returns the input image.
	It is used for debugging and has no actual use.
\item	\textbf{negative}: $p'(x,y) = 255 - p(x,y)$ -- It returns the negative of the input image.
\item	\textbf{threshold}: 
	$ p'(x,y) = \left\{
	\begin{array}{ll}
		255, 	& p(x,y) \geq th\\
		0,	& p(x,y) < th\\
	\end{array} 
	\right. 
	$ -- It replaces all pixel values above a certain threshold to maximum (255 for 8 bit
	depth) and all others to minimum (zero). Essentially, it transforms a greyscale
	image to a black and white.
\end{itemize}
None of this accelerators posed any difficulty to implementation.

\subsection{Contrast and Brightness Transformations}

The ``contrast'' accelerator adjusts the contrast and the brightness of the image.
It uses the following equation:
\[
\begin{array}{ll}
	\alpha	&= 259 \cdot (c + 255) / \left[ 255 \cdot (259 - c) \right] \\
	\beta	&= b + 128\\
	p' &= \alpha \cdot ( p - 128) + \beta\\
\end{array}
\]

As it can be seen, it uses both multiplication and division. An attempt to implement this
transform with fixed point arithmetic was fruitless, as the accelerator was not
placeable. The formula was simplified, with some precision loss, to ease the computation.

\[
\begin{array}{ll}
	\alpha &= \left[ 256 \cdot (c+256)  \right] / (256 - c) \\
	\beta &= b + 128\\
	p' &= \left[ \alpha \cdot (p - 128) + \beta \right] / 256 + \beta\\
\end{array}
\]

This simplification allowed the use of integer arithmetic, while multiplication
and division by 256 were done with shift operations. This version is placeable,
however it is very sensitive in routing. 
Changing the width of $\alpha$ and $\beta$ by a couple of bits
may render the accelerator from achieving timing closure to fail with unrouted nets.
Reducing $b$ and $c$ to byte sized operands cause the simplification of the divider
from 32b to 8b, which reduced core LUT6 utilization by more than 30\%, ensuring
successful routing and timing closure.

However, despite that a \gls{pblock} LUT6 utilization of only 50\% was achieved
for the accelerator core count oriented design,
this core remained very sensitive at routing with respect to the \glspl{pp} selection.
For this reason, it was chosen to implement the initial configuration,
despite the presence of much bigger cores, like the Gaussian Blur accelerator
which reaches an 80\% utilization.

\subsection{The Sharpen, Emboss and Outline Filters}

These three accelerators are grouped together as they all use a 3x3 kernel,
the computation was quite straightforward and implementation was easy.
The ``sharpen'' filter, as the name suggests, is used for sharpening the image.
It is quite often used at low intensity for improving the aesthetic result of
photographs or for compensating excessive blurrness due to out-of-focus errors,
albeit with moderate results. 
The ``emboss'' kernel is used to increase the sense of depth in an image.
The ``outline'' kernel is a simple method
of edge detection, used in computer vision applications.


\begin{figure}[H]
\centering
\begin{subfigure}[b]{.3\linewidth}
	\[
	\left[
	\begin{array}{rrr}
	0  & -1  &  0 \\
	-1 &  5  & -1 \\
	0  & -1  &  0 \\
	\end{array}
	\right]
	\]
	\caption{Sharpen}
\end{subfigure}
~
\begin{subfigure}[b]{.3\linewidth}
	\[
	\left[
	\begin{array}{rrr}
	-1  & -1  & -1 \\
	-1  &  8  & -1 \\
	-1  & -1  & -1 \\
	\end{array}
	\right]
	\]
	\caption{Outline}
\end{subfigure}
~
\begin{subfigure}[b]{.3\linewidth}
	\[
	\left[
	\begin{array}{rrr}
	-2  & -1  &  ~~~0 \\
	-1  &  1  &  ~~~1 \\
	 0  &  1  &  ~~~2 \\
	\end{array}
	\right]
	\]
	\caption{Emboss}
\end{subfigure}
\end{figure}

\subsection{The Sobel/Scharr Filter}

The Sobel filter is a filter used to emphasize local pixel differences,
for edge detection in computer vision. The filter uses two 3x3 kernels,
one for horizontal difference and one for vertical. Essentially,
there are two 2D convolutions taking place concurrently on the same dataset.
The two convolution results produced for every pixel are combined to
form the output pixel. After the original kernel by Sobel and Feldman,
there were subsequent attempts to optimize its properties. A common variant,
by Scharr, is also implemented within the same accelerator and is user selectable at run-time.

\begin{figure}[H]
\centering
\begin{subfigure}[b]{.45\linewidth}
	\[
	h_x = \left[
	\begin{array}{rrr}
	1 & 0 & -1 \\
	2 & 0 & -2 \\
	1 & 0 & -1 \\
	\end{array}
	\right]
	\]
	\caption{Horizontal Difference Filter}
\end{subfigure}
~
\begin{subfigure}[b]{.45\linewidth}
	\[
	h_y = \left[
	\begin{array}{rrr}
	1 & 2 & 1 \\
	0 & 0 & 0 \\
	-1 & -2 & -1\\
	\end{array}
	\right]
	\]
	\caption{Vertical Difference Filter}
\end{subfigure}
\caption{The Sobel kernel}
\end{figure}



\begin{figure}[H]
\centering
\begin{subfigure}[b]{.45\linewidth}
	\[
	h_x = \left[
	\begin{array}{rrr}
	3  & 0 & -3 \\
	10 & 0 & -10 \\
	3  & 0 & -3 \\
	\end{array}
	\right]
	\]
	\caption{Horizontal Difference Filter}
\end{subfigure}
~
\begin{subfigure}[b]{.45\linewidth}
	\[
	h_y = \left[
	\begin{array}{rrr}
	3 & 10 & 3 \\
	0 & 0 & 0 \\
	-3 & -10 & -3\\
	\end{array}
	\right]
	\]
	\caption{Vertical Difference Filter}
\end{subfigure}
\caption{The Scharr kernel}
\end{figure}

The final pixel that represents the gradient at that spatial point
is calculated by combing $p'_x$ and $p'_y$ as:

\[	p' = \sqrt{p_x^{'2} + p_y^{'2}} \approx |p'_x| + |p'_y|  \]

The approximation above is quite frequently used even for software implementations.
It is also used in this accelerator of course, as the space of 800 LUT6 would not
permit the implementation of square root calculation. The visual difference is negligible.

The implementation was made possible only by forcing the HLS to use DSP48 tiles
to perform the multiplication. For this reason, it is compatible only with the 
``big'' \glspl{rp} of the 16-accelerator design.

\subsection{The Gaussian Blur Filter}

The Gaussian Blur filter is an image smoothing filter.
It is used for various purposes, mostly for noise suppression.
The kernel derives from the two-dimensional Gaussian function:

\[ G(x,y) = \frac{1}{2\pi\sigma^2} e^{-\frac{x^2 + y^2}{2\sigma^2}}	\]

The $\sigma$ is the standard deviation of the Gaussian distribution and effectively
controls the intensity of the blurring. For our accelerator, we attempted to implement
the following kernel:

\[
\left[
\begin{array}{rrrrr}
1  &   4  &  6  &  4  &  1 \\
4  &  16  &  24 & 16  &  4 \\
6  &  24  &  36 & 24  &  6 \\
4  &  16  &  24 & 16  &  4 \\
1  &   4  &  6  &  4  &  1 \\
\end{array}
\right]
\]

Note that the sum of all weights are 256, and the final pixel will be divided by this
value to avoid saturation and maintain image brightness.

As in Sobel, the use of DSP48 tiles were essential in order to fit the accelerator in 800 LUT6s.
The initial attempts were unsuccessful, until it was discovered by pure chance 
that the following kernel is implementable to our requirements:

\[
\left[
\begin{array}{rrrrr}
1  &   4  &  6  &  4  &  1 \\
4  &  16  &  25 & 16  &  4 \\
6  &  25  &  32 & 25  &  6 \\
4  &  16  &  25 & 16  &  4 \\
1  &   4  &  6  &  4  &  1 \\
\end{array}
\right]
\]

Note that the central term was reduced to 32, and the difference of four was spread equally
to its four perpendicular neighbors. Without any other change in source code or
synthesizer settings, the estimated utilization was reduced by 11\% for \glspl{lut}
and 9\% for flip-flops, most probably due to increased symmetry.

\subsection{Resource Utilization and Latency}

An effort is made to quantify the resource utilization of the acceleration cores,
as well as to obtain a primary performance indicator. In table \ref{tab:acc-util}
we can see the results coming from two sources: The latency is an estimation
which is provided after compiling the HLS code to HDL. The resource utilization
may be also provided post-compilation, however it appears the estimation
is highly inaccurate and a post-synthesis result is presented, after exporting the core
to a .dcp file. A comparison between resource utilization and expected latency
is displayed in figure \ref{fig:acc-eff}. As we see, accelerator size increases
almost linearly with data width whereas latency gain is linear only at pixel transformations.
On accelerators performing 2D convolution the gain decreases with the increase of width.
This can be attributed to a sub-optimal BRAM access by the code HLS generates.

\begin{figure}[htb!]
\centering
\resizebox{\textwidth}{!}{%
\begin{tabular}{cl rrrr rrrrrrr}
\toprule
&	& \rotatebox{90}{loopback} & \rotatebox{90}{negative} & \rotatebox{90}{threshold}
	& \rotatebox{90}{contrast} & \rotatebox{90}{sharpen}  & \rotatebox{90}{emboss}
	& \rotatebox{90}{outline}  & \rotatebox{90}{gauss}    & \rotatebox{90}{gauss/dsp}
	& \rotatebox{90}{sobel}    & \rotatebox{90}{sobel/dsp}\\
\cmidrule(l{.3em}r{.3em}){1-13}
\multirow{5}{*}{\rotatebox{90}{8 bit}} &
LUT	& 125 & 110 & 111 & 288 & 300 & 326& 346 & 587 & 466 & 454 & 635 \\
&FF	& 142 & 114 & 111 & 303 & 290 & 357& 394 & 553 & 502 & 492 & 628 \\
&DSP	& 0   & 0   & 0   & 0   & 0   & 0  & 0   & 4   & 8   & 0   & 2   \\
&BRAM	& 0   & 0   & 0   & 0   & 3   & 3  & 3   & 5   & 5   & 3   & 3   \\
&Latency& 2.07& 2.07& 2.07& 2.07&4.15 &8.29& 8.29& 6.22& 6.22& 4.15& 4.15\\
\cmidrule(l{.3em}r{.3em}){1-13}
\multirow{5}{*}{\rotatebox{90}{16 bit}} &
LUT	& 147 & 122 & 136 & 420 & 423 & 429 & 507 & 887 & 770& 706 & 1048\\
&FF	& 182 & 154 & 137 & 398 & 386 & 443 & 578 & 942 & 741& 765 & 970 \\
&DSP	& 0   & 0   & 0   & 0   & 0   & 0   & 0   & 8   & 16 & 0   & 4   \\
&BRAM	& 0   & 0   & 0   & 0   & 3   & 3   & 3   & 5   & 5  & 3   & 3   \\
&Latency& 1.04& 1.04& 1.04& 1.04&3.11 & 6.22& 6.22&4.15 &4.15&3.11 &3.11 \\
\cmidrule(l{.3em}r{.3em}){1-13}
\multirow{5}{*}{\rotatebox{90}{32 bit}} &
LUT	& 171 & 146 & 166 & 680 & 598 & 662 & 722 &1600 &1339&1268& 1968 \\
&FF	& 262 & 234 & 189 & 561 & 620 & 702 & 873 &1658 &1267&1294& 1950 \\
&DSP	& 0   & 0   & 0   & 0   & 0   & 0   & 0   &12   &28  & 0  & 8    \\
&BRAM	& 0   & 0   & 0   & 0   & 3   & 3   & 3   &5    & 5  & 3  & 3    \\
&Latency& 0.52& 0.52& 0.52& 0.52& 2.59& 5.18& 5.18&3.11 &3.11&2.59&2.59  \\
\cmidrule(l{.3em}r{.3em}){1-13}
\multirow{5}{*}{\rotatebox{90}{64 bit}} &
LUT	& 231 & 206 & 238 &1206 & 1057&1235 &1362 &3175 &2541&2291& 3784 \\
&FF	& 422 & 394 & 293 & 907 & 1099&1342 &1608 &3168 &2585&2413& 3559 \\
&DSP	& 0   & 0   & 0   & 0   & 0   & 0   & 0   & 20  &52  &0   & 16   \\
&BRAM	& 0   & 0   & 0   & 0   & 3   & 3   & 3   & 5   & 5  &3   & 3    \\
&Latency& 0.26& 0.26& 0.26& 0.26& 2.33& 4.66& 4.66& 2.59&2.59&2.33& 2.33 \\
\bottomrule
\end{tabular}
}
\caption{Accelerator Resource Utilization (post-synthesis) and Latency (HLS estimation).\\
Target FPGA was Z-7020 at 7.5ns. BRAM is in 18k units and latency is measured in millions of cycles.\\
The ``/dsp'' suffix denotes that the integer multiplication was assigned manually to a DSP48 core.}
\label{tab:acc-util}
\end{figure}

\begin{figure}[htb!]
\begin{gnuplot}[terminal=pdf,terminaloptions=color]
set grid
set key inside top left
set xlabel 'Accelerator Data Width'
#set xtics rotate by -25 offset -1.2,-0.2
set ylabel 'LUT utilization'
set y2label 'Core Latency'
set xrange [-0.5:3.5]
set yrange [0:4000]
set y2range [0:9]

set style line 1 lc rgb '#f0f0f0' lt 1 lw 1 pt 7 pi -1 ps 0.5
set style line 2 lc rgb '#e0e0e0' lt 1 lw 1 pt 7 pi -1 ps 0.5
set style line 3 lc rgb '#d0d0d0' lt 1 lw 1 pt 7 pi -1 ps 0.5
set style line 4 lc rgb '#c0c0c0' lt 1 lw 1 pt 7 pi -1 ps 0.5
set style line 5 lc rgb '#b0b0b0' lt 1 lw 1 pt 7 pi -1 ps 0.5
set style line 6 lc rgb '#a0a0a0' lt 1 lw 1 pt 7 pi -1 ps 0.5
set style line 7 lc rgb '#909090' lt 1 lw 1 pt 7 pi -1 ps 0.5
set style line 8 lc rgb '#808080' lt 1 lw 1 pt 7 pi -1 ps 0.5
set style line 9 lc rgb '#707070' lt 1 lw 1 pt 7 pi -1 ps 0.5
set pointintervalbox 1

set style line 11 lc rgb '#ffffff' lt 1 dt 10 lw 1 pt 5 pi -1 ps 0.4
set style line 12 lc rgb '#e8e8e8' lt 1 dt 10 lw 1 pt 5 pi -1 ps 0.4
set style line 13 lc rgb '#d0d0d0' lt 1 dt 10 lw 1 pt 5 pi -1 ps 0.4
set style line 14 lc rgb '#c0c0c0' lt 1 dt 10 lw 1 pt 5 pi -1 ps 0.4
set style line 15 lc rgb '#b0b0b0' lt 1 dt 10 lw 1 pt 5 pi -1 ps 0.4
set style line 16 lc rgb '#a0a0a0' lt 1 dt 10 lw 1 pt 5 pi -1 ps 0.4
set style line 17 lc rgb '#909090' lt 1 dt 10 lw 1 pt 5 pi -1 ps 0.4
set style line 18 lc rgb '#808080' lt 1 dt 10 lw 1 pt 5 pi -1 ps 0.4
set style line 19 lc rgb '#707070' lt 1 dt 10 lw 1 pt 5 pi -1 ps 0.4


set style data linespoints

plot "data/acc-util.plot" using 2:xtic(1) title "loopback" ls 1,\
			"" using 3 title "negative" ls 2,	\
			"" using 4 title "threshold" ls 3,	\
			"" using 5 title "contrast" ls 4,	\
			"" using 6 title "sharpen" ls 5,	\
			"" using 7 title "emboss" ls 6,		\
			"" using 8 title "outline" ls 7,	\
			"" using 9 title "gauss" ls 8,		\
			"" using 11 title "sobel" ls 9,		\
	"data/acc-perf.plot" using 2:xtic(1) notitle ls 11 axes x1y2,	\
			"" using 3 notitle ls 12 axes x1y2, \
			"" using 4 notitle ls 13 axes x1y2, \
			"" using 5 notitle ls 14 axes x1y2, \
			"" using 6 notitle ls 15 axes x1y2, \
			"" using 7 notitle ls 16 axes x1y2, \
			"" using 8 notitle ls 17 axes x1y2, \
			"" using 9 notitle ls 18 axes x1y2, \
			"" using 11 notitle ls 19 axes x1y2
\end{gnuplot}
\caption{Resource Utilization vs. Core Latency}
\label{fig:acc-eff}
\end{figure}

\section{Accelerator Interface}
\label{sec:accelerator-interface}

Each accelerator connects to the system via two pathways -- one \gls{axistream} that 
exchanges data with the AXI DMA using two channels, a read for incoming data and
a write for outgoing, and one \gls{axilite} for status and control.
As already discussed in section \ref{sec:pr-workflow}, the \gls{rm} must remain
consistent throughout all implementations. In Vivado HLS this translates to:

\begin{itemize}
\item	The width of the \gls{axistream} interface must remain consistent throughout \glspl{rm}.
\item	All \glspl{rm} will have exactly one read and one write channel. 
\item	The address width of \gls{axilite} interface must remain consistent for all \glspl{rm}.
\item	The implementation tools expect the module name to be consistent throughout
	all \gls{rm} implementations.
\end{itemize}

The width of the data stream is the most important design choice for the accelerator performance.
Our system does not care for the value and the kernel driver does not see it; still,
it must remain consistent throughout all \gls{rm} variants of each \gls{rp}. The \glspl{rp} themselves
however, do not need to match each other.

In our application, we made the accelerator width a user configurable parameter. 
The script that builds the \gls{hls} modules (see appendix \ref{apx:script-hls}) 
will generate a C header file that is included by all \gls{hls} C++ source files
and contains the appropriate definitions to allow a 8b, 16, 32b or 64b accelerator width.
Higher widths are possible but will require source modifications.

The accelerator will take advantage of wider stream by loading two, four or eight pixel at once,
and process them in parallel. Whether the \gls{hls} synthesizer does this efficiently and 
if performance scales linearly, it will have to be determined by benchmarking.

Regarding the number of channels, it was decided to be two. This was an obvious choice,
given the exact mapping with AXI DMA channels -- one receive, one trasmit. 
However, it is still the simplest choice and a lot more sophisticated architectures
can be built. For a discussion on this, see section \emph{TODO ref για future work}.

As for the \gls{axilite} channel, its width is strictly 32 bits, as defined by the protocol.
However, the address width is variable and depends on the addressable space of the accelerator.
This is decided indirectly by the mapping of the accelerator input and output variables.

In order to demonstrate how all this is done, in listing \ref{lst:func-declaration},
a \gls{rm} function declaration is displayed, among with its \gls{hls} directives.
\vspace*{\fill}
\pagebreak
\begin{figure}[H]
\centering
\begin{lstlisting}[style=basic,language=C]
int CORE_NAME(axi_stream_t& src, axi_stream_t& dst, int brightness, int contrast)
{
#pragma HLS INTERFACE axis port=src bundle=INPUT_STREAM
#pragma HLS INTERFACE axis port=dst bundle=OUTPUT_STREAM
#ifdef CLK_AXILITE
#  pragma HLS INTERFACE s_axilite clock=axi_lite_clk port=brightness bundle=control offset=0x10
#  pragma HLS INTERFACE s_axilite clock=axi_lite_clk port=contrast bundle=control offset=0x18
#  pragma HLS INTERFACE s_axilite clock=axi_lite_clk port=return bundle=control offset=0x38
#else
#  pragma HLS INTERFACE s_axilite port=brightness bundle=control offset=0x10
#  pragma HLS INTERFACE s_axilite port=contrast bundle=control offset=0x18
#  pragma HLS INTERFACE s_axilite port=return bundle=control offset=0x38
#endif
#pragma HLS INTERFACE ap_stable port=brightness
#pragma HLS INTERFACE ap_stable port=contrast
/* module implementation */
}
\end{lstlisting}
\caption{An \gls{hls} \gls{rm} function declaration.}
\label{lst:func-declaration}
\end{figure}
It is an example from the ``\texttt{contrast}'' accelerator. 
The first thing to notice is that the function name does not 
mention the \gls{rm}'s variant and it is a generated constant. 
Its contents are actually related to the ``\texttt{axi\_stream\_t}'' type 
which is a user defined type. A few lines of code are worth a thousand words of comments:

\begin{figure}[H]
\centering
\begin{lstlisting}[style=basic,language=C]
/* generated.h */
#define CORE_NAME zcore16
typedef uint16_t axi_data_t;
/* ... */

/* common.h */
typedef ap_axiu<sizeof(axi_data_t)*8, 1, 1, 1>  axi_elem_t;
typedef hls::stream<axi_elem_t> axi_stream_t;
/* ... */
\end{lstlisting}
%\caption{The \gls{rm} data width code.}
\end{figure}

The file ``\texttt{generated.h}'' is generated by the script shown in appendix \ref{apx:script-hls}.
The ``\texttt{axi\_data\_t}'' is essentially our accelerator word size, and we see, it is used
to define ``\texttt{axi\_elem\_t} which describes the full \gls{axistream} signal set.
From left to right, it defines the effective data (TDATA) width, the user-defined (TUSER)
width, the master/source identity (TID) and the slave/sink identity (TDEST). All quantities are
in bits and names in parenthesis express the standard's nomenclature. 
The ``\texttt{axi\_stream\_t}'' represents the \gls{axi} stream itself 
and it is a standard C++ IO stream. The implemented accelerator functionality, 
i.e. the contrast and brightness adjustment, is not represented in the function name as it
will be the name of the \gls{rm} in design hierarchy, not solely of this variant. All variants
have to share the same name. However, the word width is represented -- this is to allow the
system to implement \glspl{rp} of varying sizes. However, no such an implementation was done
in this work.

Back to our listing, we can now discuss the \gls{axilite} status/control port.
Every such an input must be mentioned in three places:
\begin{enumerate}
\item	At function declaration. If the variable is an output, it must be declared as a pointer.
\item	At \texttt{HLS INTERFACE s\_axi\_lite} pragma, where each variable is defined
	as a member of a specific \gls{axilite} interface (``\texttt{port}'' and ``\texttt{bundle}''
	properties) and is assigned an address (``\texttt{offset}'' property). It is called an
	offset as its value represents the address of this variable in respect to the start
	of this \gls{axilite} interface's address space.
\item	At \texttt{HLS INTERFACE ap\_stable}. This forms an explicit promise on our part
	that the bus master will not modify the contents of these variables during module execution.
\end{enumerate}

It is the value of the ``\texttt{offset}'' property that actually defines the address width
of the \gls{axilite} interface, i.e. the size of \texttt{AWADDR} for the write channel
and \texttt{ARADDR} for the read channel. Their size will be the minimum sufficient
to accommodate the maximum offset among all memory mapped registers that represent the variables.
The compiler will not differentiate between read-only or write-only variables, so a safe way
to impose the width of both channels is to place the \texttt{return} variable at the end of
the addressable space. Here, by placing \texttt{return} at 0x38 (variables are aligned at 8 bytes
boundaries) we are defining a space of 0x00..0x3F, which corresponds to 6 bits of address width.
Within this address space, the first 16 bytes are reserved by HLS. The first accelerator
argument is placed at 0x10, and subsequent arguments at 8 bytes distance between them.
This convention may be altered but a header file (\texttt{param.h}) modification
of the kernel driver is necessary.

The property ``\texttt{clock}'' was made optional and it allows the \gls{axilite}
to be clocked by a different source from the input and output streams. 
This is because generally \gls{axilite} slaves achieve lower frequency than \gls{axistream} ones
and it might be worthy to explore if higher clock can be achieved by using separate clock domain.

Apart from the explicit memory mapped registers, some implicit are instantiated 
that form the status and control protocol of the \gls{hls} core, the default signal
set being the \texttt{ap\_ctrl\_hs}. The programming sequence consists of the steps
described in the following list, and an example memory map is given in figure \ref{fig:hls-map}.

\begin{enumerate}
\item	The signal \texttt{ap\_idle} should be asserted, indicating the core is idling.
\item	All memory mapped registers corresponding to core parameters must be written now.
\item	The user asserts \texttt{ap\_start}, commencing the execution. In response,
	the core will de-assert \texttt{ap\_idle}.
\item	When all input is read, the \texttt{ap\_ready} is asserted, and when all
	output is written, the \texttt{ap\_done} is also asserted.
	The return value may now be read.
\item	The \texttt{ap\_idle} is asserted, indicating that the core has
	finished execution and is now idle.
\end{enumerate}

\begin{figure}[tb]
\scalebox{1.0}{
\centering
  \newcommand{\bitlabel}[2]{%
  \bitbox[]{#1}{%
    \raisebox{-10pt}[4ex][0pt]{%
      \turnbox{45}{\fontsize{7}{7}\selectfont#2}%
    }%
  }%
}
\begin{bytefield}[rightcurly=.,bitwidth=1.1em,endianness=big,bitformatting=\footnotesize\ttfamily,
	boxformatting={\centering\ttfamily},bitwidth=0.75em]{32}
\bitlabel{24}{}
\bitlabel{1}{auto\_restart}
\bitlabel{1}{reserved}
\bitlabel{1}{reserved}
\bitlabel{1}{reserved}
\bitlabel{1}{ap\_ready}
\bitlabel{1}{ap\_idle}
\bitlabel{1}{ap\_done}
\bitlabel{1}{ap\_start}\\
\bitheader{8,31}\\
\begin{leftwordgroup}{\rotatebox{90}{HLS reserved}}
\begin{rightwordgroup}{0x00}
\bitbox{24}{ \hspace{7em} Control/Status }
\bitboxes{1}{~}
\bitboxes{1}{~}
\bitboxes{1}{~}
\bitboxes{1}{~}
\bitboxes{1}{~}
\bitboxes{1}{~}
\bitboxes{1}{~}
\bitboxes{1}{~}
\end{rightwordgroup} \\
\begin{rightwordgroup}{0x04}
\bitbox{31}{ \hspace{1em} Global Interrupt Enable } 
\bitbox{1}{\tiny G\\I\\E}
\end{rightwordgroup}\\
\begin{rightwordgroup}{0x08}
\bitbox{30}{ \hspace{2em} Core Interrupt Enable } 
\bitbox{1}{\rotatebox{90}{\tiny ready}}
\bitbox{1}{\rotatebox{90}{\tiny done}}
\end{rightwordgroup} \\
\begin{rightwordgroup}{0x0C}
\bitbox{30}{ \hspace{2em} Core Interrupt Status } 
\bitbox{1}{\rotatebox{90}{\tiny ready}}
\bitbox{1}{\rotatebox{90}{\tiny done}}
\end{rightwordgroup}
\end{leftwordgroup}\\
\begin{rightwordgroup}{0x10}
\bitbox{24}{ \hspace{7em}\rmfamily\itshape reserved }
\bitbox{8}{ \bfseries brightness }
\end{rightwordgroup}\\
\begin{rightwordgroup}{0x14}
\wordbox{1}{ \rmfamily\itshape reserved }
\end{rightwordgroup}\\
\begin{rightwordgroup}{0x18}
\bitbox{24}{ \hspace{6.75em} \rmfamily\itshape reserved }
\bitbox{8}{ \bfseries contrast }
\end{rightwordgroup}\\
\begin{rightwordgroup}{0x1C}
\wordbox{1}{ \rmfamily\itshape reserved }
\end{rightwordgroup}\\
\wordbox[lrt]{1}{\rmfamily\itshape unused} \\
\skippedwords \\
\wordbox[lrb]{1}{} \\
\begin{rightwordgroup}{0x38}
\wordbox{1}{ \bfseries return }
\end{rightwordgroup}\\
\begin{rightwordgroup}{0x3C}
\wordbox{1}{ \rmfamily\itshape unused }
\end{rightwordgroup}
\end{bytefield}
}
\caption{The ``contrast'' accelerator address space.}
\label{fig:hls-map}
\end{figure}



