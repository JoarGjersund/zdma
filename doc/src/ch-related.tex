\chapter{Related Work}

The concept of allowing a user software application an abstracted interface to the programmable logic on an FPGA is not new.

The RIFFA Framework \cite{riffa} abstracts the data transfer between a user application and a PCI-e attached FPGA accelerator
by the means of named pipes. It is minimalistic by design but significant effort was put to offer high performance
and to make it compatible with several common FPGA development boards from both Xilinx and Altera/Intel. They support the use of
multiple FPGAs but not a direct FPGA-to-FPGA link. Partial reconfiguration is not supported. It is the only open-source 
framework that provides support for Microsoft Windows.

The EPEE \cite{epee} and the ffLink \cite{fflink} offer similar set of features. 
The former has support for interrupt acknowledgement whereas the latter supports PCI-e 3.0.

A significant differentiation is DyRACT \cite{dyract}. It is the first to support partial reconfiguration. However it is the only
that does not support zero-copy DMA.

The newer JetStream \cite{jetstream} is a major step-up, offering every desirable feature: Partial Reconfiguration, Zero Copy,
multiple board support and direct FPGA-to-FPGA link. It also supports PCI-e 3.0 and achieves strong performance results.

All the aforementioned works are academic and open-sourced.
One of the most notable is from Xillybus.
It is lightweight in FPGA resources but lacks most features discussed.
It does not support neither partial reconfiguration nor zero-copy DMA. 
It offers multiple board support but no direct FPGA-to-FPGA link. 
It comes with a price tag of 25K to 75K USD and no royalties.
A Linux kernel driver is offered in open source and Windows is also supported.


Another important proprietary work is from Nortwest Logic. Not many details are available for the internals of this product
but the vendor asserts support even for the PCI-e 4.0. It offers high performance but very high FPGA resource utilization.
No price is published.


The work presented in this thesis has a key differentiation point that it uses AXI as a data transport instead of PCI-e,
as it focuses on FPGA SoCs and not on add-on FPGA cards. To our knowledge, there is currently no other work that is based
on AXI, except for Xillybus which has a Microblaze port which uses AXI for transport which nonetheless is being phased-out.
Feature-wise, we support all the (non PCI-e related) features offered by all the open-source works except for JetStream.
Compared to JetStream, this work lacks support for direct FPGA-to-FPGA link. 
Finally, since we support zero-copy DMA we found no reason to implement a buffered one. 
However, EPEE, ffLink and JetStream, chose to implement both transfer types.
