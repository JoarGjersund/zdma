\chapter{Software Framework}

One of the primary goals of this work is to offer an abstraction layer that hides
all the hardware details from the end user. The software framework acts as an intermediary
that on the one end orchestrates all control and communication of the FPGA part, while
on the other end offers a simple software API ready to be used by a programmer that needs not
to have any understanding of the underlying hardware.

The software framework consists of two parts. The major work is done by a kernel driver.
It is responsible of the coordination of all system elements from receiving the command
to process data to the delivery of the results. However, the kernel driver communicates
to the userspace with the means of system calls and I/O control commands, which is not
an appropriate interface for the end user. A separate system library was implemented,
which is mostly used either as a wrapper, or to perform tasks that cannot (or should not)
be done in kernel mode, namely the file operations to read the partial bitstreams.

\section{Communicating with the Hardware}

One of the most frequent reasons for implementing something in kernel space,
is the need to communicate directly to the hardware. In processor architectures
that support hierarchical protection domains, the hardware is exposed only to
code running in a privileged mode, typically the operating system.
If the user application requires access to a device, it has to call the operating
system to execute on its behalf. This way, the system enforces security policies
and offers hardware abstraction. 

Since for our work we do need to manipulate hardware directly, we have to do it from
within the kernel. Programming the kernel however, is a challenging feat.
It is a completely different enviornment, lacking all the tools and libraries any
user is familiar with -- including the standard C library. Many everyday operations
that are taken for granted, like file I/O or floating point arithmetic,
are not allowed or at least greatly discouraged. Debugging is much more restricted
and complicated while many of the system safety nets are not present -- a programming
error is not isolated at the running process and its resources, but may affect the whole system.
Last but not least, modifying kernel code is not as direct as it is in the userspace.

A naive workaround at these apparent difficulties is offered though the \texttt{/dev/mem} Linux device. 
It consists a direct interface to the machine's physical address space and it can be mapped to a user virtual space. 
Its primary use is for debugging the system and is always disabled in production systems, but many hardware engineers
find it convenient since it allows them to program a Linux based FPGA SoC as if it was a bare-metal environment,
with the only restriction that interrupts cannot be detected. 
Nonetheless, this solution was not considered, as it negates the operating system's raison d'être --
security and hardware abstraction.

A much more ``clean'' solution is given by the Linux Userspace I/O system. Written with industrial I/O cards
in mind, this system allows the control of an interrupt and memory capable device using only
a minimal device driver that only declares the device's hardware resources. All control and data processing
can be done at the userspace using the tools and libraries the programmer is familiar with. 
The device memory resources can be mapped to the virtual address space
and its interrupts may be detected with the use of blocking \texttt{read} or the \texttt{select} system call.
This solution can be made even more attractive from the fact that Vivado HLS is able to generate
the aforementioned minimal device driver automatically during IP packaging.
However, UIO may be ideal for simple I/O devices but it is too restrictive and inflexible for anything more complicated.
Since the control logic is written in userspace, the programmer loses all access to kernel facilities
and data structures that are usually necessary for more complex tasks involving other kernel subsystems.

At the end, we opted for implementing our core system entirely in-kernel.

\subsection{Using the AXI DMA cores}

In a bare-metal environment one would program a PL peripheral by manipulating its control / status register.
This is also entirely possible in an operating system, and some times it might be the only way.
However it has two major downsides. The first is that it sacrifices portability, 
as the programming sequence of an IP core is fundamentally dependent on the specific hardware.
A newer core revision, a different core configuration, 
a different host platform, or even worse, an IP core from a different
provider, may have significantly different programming interface. For the designer, this means that
they have to learn the low-level programming details of the IP core, and in case it needs to be
updated or replaced, they must put a significant effort to port the software to the new version.
The other downside is that direct manipulation of hardware control registers bypasses all OS
subsystems that may offer useful support functionality and integration with other kernel services.

The Linux kernel supports myriads of DMA controllers on the several computing platforms it is ported.
Likewise, there are several kernel subsystems that wish to use these DMA controllers.
In order to offer abstraction, the ``Linux DMA API'' is offered. It is a stable API that provides
functions 







